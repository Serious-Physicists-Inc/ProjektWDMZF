\documentclass[a4paper,12pt]{article}
\usepackage[utf8]{inputenc}
\usepackage[T1]{fontenc}
\usepackage{indentfirst}
\usepackage{polski}
\usepackage{graphicx}
\usepackage[margin=2.5cm]{geometry}
\usepackage{float}
\usepackage{amsmath}
\usepackage{amssymb}
\usepackage{array}

\title{Dokumentacja projektu}
\author{Mikołaj Suszek, Hubert Rączkiewicz}
\date{8 grudnia 2025}

\begin{document}
\maketitle
\section{Wstęp}
Odkryte przez Erwina Schrödingera równanie falowe legło u podstaw nowego działu fizyki – mechaniki kwantowej. Dokonany przełom znalazł zastsowanie w wielu dziedzinach nauk ścisłych. 
\section{Cel projektu}
Celem projektu jest utworzenie interaktywnej animacji wizualizującej poszczególne stany energetyczne w atomie wodoru, zarówno dla ustalonych liczb kwantowych jak i stanów przejściowych. Pozwoli to na lepsze zrozumienie konfiguracji elektronowych i wynikających z nich właściwości chemicznych i fizycznych wodoru.
\section{Opis modelowania zjawiska fizycznego}
Mianem chmury elektronowej w atomie określa się zagęszczenie  prawdopodobieństwa znalezienia elektronu w danym punkcie w przestrzeni. Jej kształt uzależniony jest od trzech liczb kwantowych: głównej, pobocznej i magnetycznej. Przy czym, liczby główna i poboczna wpływają na radialny rozkład prawdopodobieństwa, zaś na rozkład kątowy określają liczby poboczna i magnetyczna. Ilościowo prawdopodobieństwo znalezienia elektronu w danym punkcie opisuje kwadrat modułu funkcji falowej elektronu. Istotą problemu jest analityczne lub numeryczne wyznaczenie funkcji falowej, będącej rozwiązaniem równania Schrödingera. Obecnie, równania Schrödingera zostało rozwiązane analitycznie wyłącznie dla atomu wodoru. Otrzymana funkcja falowa dana jest wzorem:
\begin{equation}
     \Psi(r,\theta,\varphi,t)= r^l\left ( \frac{2}{na_0}\right )^{l+1}L_{n-l-1}^{2l+1}\left (\frac{2r}{na_0}\right )e^{-r/na_0} \cdot(-1)^m\sqrt{\frac{(2l+1)(l-m)!}{4\pi(l+m)!}}P_{l,m}(\cos\theta) e^{im\varphi}\cdot e^{\frac{iEt}{\hslash}}
\end{equation}
\section{Opis wykorzystywanych narzędzi}
wersja języka Python: 3.9, IDE: Pycharm, wstępnie wytypowane do użycia biblioteki: NumPy, SymPy, SciPy, MatplotLib, PyVista
\section{Ogólny opis projektu i możliwe alternatywy}
Elementy bazowe:
\begin{itemize}
    \item Moduł obliczający wartości funkcji falowej atomu wodoru w stanie stacjonarnym dla danych liczb kwantowych i na ich podstawie wyznaczający rozkład prawdopodobieństwa znalezienia elektronu.
    \item Moduł generujący siatkę przestrzeni do wizualizacji wartości.
    \item Zestaw funkcji pomocniczych m.in. konwertujących wartości między układami współrzędnych przestrzennych.
    \item Interaktywny moduł wizualizujący obliczone wartości.
\end{itemize}
Elementy dodatkowe:
\begin{itemize}
    \item Moduł obliczający wartości funkcji falowej atomu wodoru w superpozycji stanów energetycznych (w trakcie przejść elektronowych).
    \item Zestaw funkcji optymalizujących szybkość działania programu.
    \item Plik wykonywalny uruchamiający program poza konsolą.
\end{itemize}
\section{Specyficzne wymagania}
a) wymagania funkcjonalne:
\begin{itemize}
    \item Program oblicza wartości funkcji falowej atomu wodoru dla danej siatki przestrzennej oraz określonych liczb kwantowych.
    \item Na podstawie wcześniej wyliczonych wartości funkcji falowej wyznacza lokalną gęstość prawdopodobieństwa znalezienia elektronu w punkcie przestrzeni.
    \item Wizualizuje gęstość prawdopodobieństwa na trójwymiarowym wykresie. 
\end{itemize}
b) wymagania niefunkcjonalne:
\begin{itemize}
    \item Przy użyciu wielorakich metod optymalizuje czas obliczeń.
    \item Umożliwia zmianę parametrów wejściowych z poziomu programu.
    \item Pakuje zawarte funkcjonalności w pojedynczy plik wykonywalny.
\end{itemize}
\section{Harmonogram pracy z zadaniami do wykonania}
\begin{table}[hbpt!!]
\centering
\begin{tabular}{| m{5cm} | m{10cm} |}
\hline
Tydzień 1 (8.12.2025 - 14.12.2025) & Opracowanie szkieletu projektu. Utworzenie repozytorium Git, poszczególnych modułów wraz z szkicem ich funkcjonalności. \\ \hline
Tydzień 2 (15.12.2025 - 4.01.2026) & Praca nad modułem wyliczającym wartości funkcji falowej dla poszczególnych liczb kwantowych atomu wodoru. Opracowanie modułu liczącego gęstość prawdopodobieństwa na podstawie wejściowych wartości funkcji falowej. \\ \hline
Tydzień 3 (5.01.2026 - 11.01.2026) & Praca nad wizualizacją danych.= Dodanie możliwości zmiany parametrów danych wejściowych w trakcie wykonywania programu. \\ \hline
Tydzień 4 (12.01.2026 - 18.01.2026) & Optymalizacja kodu. Podniesienie niezawodności na błędy numeryczne. Utworzenie pliku wykonywalnego programu. \\ \hline
\end{tabular}
\end{table}
\end{document}
